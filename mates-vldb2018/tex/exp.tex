\section{Experiments}
\label{sec:exp}
We perform a few experiments on \sgg\ to validate its efficiency and effectiveness. In the interest of space, we only present a glimpse of our experiments here. More will be discussed in an extended version.

\vspace{2pt}
First off, we validate the ``usability'' of our framework. For this aim, we design a user study with $14$ participants who are all students of Computer Science. Half of them are ``novice'' users who don't know the location, and the other half are ``experts''. Participants should fulfill a task in two different frameworks: \sgg\ and {\sc Tableau}. We choose {\sc Tableau} as our competitor as it is the most advanced off-the-shelf visualization product for analyzing spatial data. For each participant, we report a variant of time-to-insight measure, i.e., the number of times that participants interact with the frameworks before fulfilling the task. Evidently, less number of interactions is preferred as it means that the participant can reach insights faster.

\begin{table}[h]
\centering
\caption{Nb. interactions for ``novice''}
\label{tbl:novice}
\begin{tabular}{c|c|c|c|c|}
\cline{2-5}
                                       	& \textbf{T1/I1} 	& \textbf{T2/I1} 	& \textbf{T1/I2}	& \textbf{T2/I2}	\\ \hline
\multicolumn{1}{|c|}{\sgg} 				& 1.99            	& 2.38	          	& 2.00              & 2.48              \\ \hline
\multicolumn{1}{|c|}{\sc Tableau}      	& 61.3            	& 57.6             	& 67.2              & 55.7              \\ \hline
\end{tabular}
\end{table}

\begin{table}[h]
\centering
\caption{Nb. interactions for ``expert''}
\label{tbl:expert}
\begin{tabular}{c|c|c|c|c|}
\cline{2-5}
                                       	& \textbf{T1/I1} 	& \textbf{T2/I1}	& \textbf{T1/I2} & \textbf{T2/I2} \\ \hline
\multicolumn{1}{|c|}{\sgg} 				& 1.72            	& 2.09	          	& 1.70              & 2.14              \\ \hline
\multicolumn{1}{|c|}{\sc Tableau}      	& 56.7            	& 49.1             	& 54.6              & 43.4              \\ \hline
\end{tabular}
\end{table}

\vspace{2pt}
On the Airbnb dataset of Paris with 1,000 points, we define two different tasks: {\em T1: ``finding a point in a requested location''} (e.g., find a home-stay in the Trocadero area of Paris), and {\em T2: ``finding a point with a requested profile''} (e.g., find a cheap home-stay with balcony.) Participants may also begin their navigation either from {\em I1: ``close to the goal''} or {\em I2: ``far from the goal''}. 

\vspace{2pt}
In {\sc Tableau}, participants employ filtering and querying tools to reach their goals. In \sgg, participants benefit from information highlighting based on their implicit feedback. Tables \ref{tbl:novice} and \ref{tbl:expert} report the number of interactions for novice and expert participants, respectively. We observe that on average $2.067$ interactions are needed to reach a defined goal in \sgg, which is  which is $24.95$ times smaller than the average number of interactions for {\sc Tableau}'s. This shows that implicit feedback capturing is an effective mechanism which helps analysts to reach their goals faster.

\vspace{2pt}
In \sgg, expert participants need $0.35$ fewer interactions on average. Interestingly, starting points, i.e., {\em I1} and {\em I2}, do not have a huge impact on number of steps. It is potentially due to the diversity component which provides distinct options and can quickly guide analyst towards their region of interest. We also observe that the task {\em T2} is an easier task than {\em T1}. This is potentially due to the similarity component where the analyst can request options similar to what she has already observed and greedily move to her preferred regions.

\vspace{2pt}
In the second part of our experiments, we briefly discuss the performance of \sgg. We will provide details of our performance study in an extended version of this paper. We execute our algorithm on $1,000$ points of Airbnb dataset and $k=20$. We report execution time as the average of the number of participants of the experiments. In general, \sgg\ needs around 2 seconds to deliver $k$ highlights. Increasing $k$ has a linear effect on the execution time. We also observe that increasing the number of dataset points influences the overall time. The bottle neck of \sgg\ is the number of IDRs generated. Once the number of IDRs is higher, the overall time for processing the request increases. Limiting time in Algorithm \ref{algo:geoh} shows that even with restricted time limit to $300ms$, the algorithm often reaches more than $71\%$ of diversity.



