\section{Related Work}
\label{sec:rel}
The literature contains several instances of feedback exploitation to guide the analyst in further analysis steps (e.g., \cite{boley2013one}). The common approach is a top-$k$ processing methodology in order to prune the search space based on the explicit feedback and recommend a small subset of interesting results of size~$k$.  As the key challenge is unify
explicit and implicit feedback, some works~\cite{AoidhBW07,Ballatore2008,Liu:2010} propose models from both explicit and implicit feedback in order to capture user preferences. A clear distinction of \sgg\ is that it doesn't aim for pruning, but leveraging the actual data with potential interesting results that the analyst may miss due to the huge volume of spatial data. While in top-$k$ processing algorithms, analyst choices are limited to $k$, \sgg\ has a freedom of choice where highlights get seamlessly updated with new analyst choices. 

\vspace{2pt}
There exist few instances of information-highlighting methods in the literature \cite{Liang2010,Robinson2011,wongsuphasawat2016voyager,willett2007scented}. All these methods are {\em objective} and do not apply to the context of spatial guidance where user feedback is involved.  In terms of recommendation, few approaches focus on spatial dimension \cite{Bao2015,Levandoski:2012} while the context and result diversification are missing.

\vspace{2pt}
Considering approaches to better generate regions as polygons and verify is a point is inside a defined region, we analysed some works. These approaches use, in general, clustered points to create concave and convex polygons. \cite{Bevis1989} proposes an algorithm for determining if any given point P, on the surface of a sphere is located inside, outside, or along the border of an arbitrary spherical polygon. \cite{DUCKHAM2008} and  \cite{FADILI2004} propose algorithms for constructing a non-convex polygons that characterizes the shape of a set of input points in the plane. These solutions consider only the definition of shapes in their proposals. \cite{ARAMPATZIS2006} and \cite{Galton2006} propose solutions for delineation of imprecise regions, either convex or concave. The most efficient and adequate and appropriated solution for generation of  convex polygons to our solution was the Quickhull  algorithm~\cite{Barber:1996}.