\section{Related Work}
\label{sec:rel}
To the best of our knowledge, the problem of spatial information highlighting using implicit feedback has been never addressed in the literature. However, our work relates to few others in their semantics.

\vspace{3pt}
\noindent {\bf Information Highlighting.} The literature contains few instances of information highlighting approaches~\cite{Liang2010,Robinson2011,wongsuphasawat2016voyager,willett2007scented}. However, all these methods are objective, i.e., they 	assume that analyst's preferences are given as a constant input and will never change in the future. This limits their functionality for serving scenarios of exploratory analysis. The only way to fulfill ``spatial guidance'' is to consider the evolutionary and subjective nature of analyst's feedback. In \sgg, the feedback vector gets updated in time based on the implicit feedback of the analyst.

\vspace{2pt}
Online recommendation approaches can also be considered as an information highlighting approach where recommended items count as highlights. Most recommendation algorithms are space-agnostic and do not take into account the spatial information. While few approaches focus on the spatial dimension~\cite{Bao2015,Levandoski:2012}, they still lack the evolutionary feedback capturing. Moreover, most recommendation methods miss ``result diversification'', i.e., highlights may not be effective due to overlaps.

\vspace{3pt}
\noindent {\bf Feedback Capturing.} Several approaches are proposed in the state of the art for capturing different forms of feedback~\cite{bhuiyan2012interactive,xin2006discovering,dimitriadou2016aide,kamat2014distributed,omidvar2015interactive,boley2013one}. The common approach is a top-$k$ processing methodology in order to prune the search space based on the explicit feedback of the analyst and recommend a small subset of interesting results of size~$k$. A clear distinction of \sgg\ is that it doesn't aim for pruning, but leveraging the actual data with potential interesting results (i.e., highlights) that the analyst may miss due to the huge volume of spatial data. Moreover, in a typical top-$k$ processing algorithm, analyst's choices are limited to $k$. On the contrary, \sgg\ enables a freedom of choice where highlights get seamlessly updated with new analyst choices. 

\vspace{2pt}
Few works formulate fusing approaches of explicit and implicit feedbacks to better capture user preferences~\cite{AoidhBW07,Ballatore2008,Liu:2010}. \sgg\ functions purely on implicit feedback and does not require any sort of explicit signal from the analyst.

\vspace{3pt}
\noindent {\bf Region Discovery.} \sgg\ finds interesting dense regions (IDRs) in order derive analyst's implicit preferences. There exist several approaches to infer a spatial region for a given set of points \cite{Bevis1989,DUCKHAM2008,FADILI2004,ARAMPATZIS2006,Galton2006,Barber:1996}. The common approach is to cluster points in form of concave and convex polygons. In~\cite{Bevis1989}, an algorithm is proposed to verify if a given point $p$ on the surface of a sphere is located inside, outside, or along the border of an arbitrary spherical polygon. In \cite{DUCKHAM2008,FADILI2004}, a non-convex polygon is constructed from a set of input points on a plane. In \cite{ARAMPATZIS2006,Galton2006}, imprecise regions are delineated into a convex or concave polygon. In \sgg, it is important to discover regions in a way to only capture mouse move points. In case a concave polygon is constructed, the ``dents'' of such a polygon may entail points which are not necessarily in $\mathcal{M}$. In \sgg, however, we adapt Quickhull~\cite{Barber:1996}, due its simplicity, efficiency and it's natural implementation of convex polygons.

