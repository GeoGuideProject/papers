% THIS IS AN EXAMPLE DOCUMENT FOR VLDB 2012
% based on ACM SIGPROC-SP.TEX VERSION 2.7
% Modified by  Gerald Weber <gerald@cs.auckland.ac.nz>
% Removed the requirement to include *bbl file in here. (AhmetSacan, Sep2012)
% Fixed the equation on page 3 to prevent line overflow. (AhmetSacan, Sep2012)

\documentclass{vldb}
\usepackage{graphicx}
\usepackage{balance}  % for  \balance command ON LAST PAGE  (only there!)
\usepackage{amssymb}
\usepackage{graphicx}
\usepackage{float}
\usepackage{subfigure}
\usepackage{mathtools}
\usepackage{eurosym}

\usepackage{pgfplots}
\usepackage{url}
\usepackage{enumitem}
\usepackage[linesnumbered,ruled]{algorithm2e}
\usepackage[export]{adjustbox}
\usepackage{xspace}
\usepackage{breqn}

\newcommand{\sgg}{{\sc PolyGuide}}

% Include information below and uncomment for camera ready
\vldbTitle{Towards Implicit Feedback Capturing on Spatial Data}
\vldbAuthors{Behrooz Omidvar-Tehrani, Pl\'acido A. Souza Neto, Tiago Oliveira Lisboa, Francisco B. Silva Junior, Felipe F. Pontes}
\vldbDOI{https://doi.org/TBD}

\begin{document}

% ****************** TITLE ****************************************

\title{Towards Implicit Feedback Capturing on Spatial Data}


% possible, but not really needed or used for PVLDB:
%\subtitle{[Extended Abstract]
%\titlenote{A full version of this paper is available as\textit{Author's Guide to Preparing ACM SIG Proceedings Using \LaTeX$2_\epsilon$\ and BibTeX} at \texttt{www.acm.org/eaddress.htm}}}

% ****************** AUTHORS **************************************

% You need the command \numberofauthors to handle the 'placement
% and alignment' of the authors beneath the title.
%
% For aesthetic reasons, we recommend 'three authors at a time'
% i.e. three 'name/affiliation blocks' be placed beneath the title.
%
% NOTE: You are NOT restricted in how many 'rows' of
% "name/affiliations" may appear. We just ask that you restrict
% the number of 'columns' to three.
%
% Because of the available 'opening page real-estate'
% we ask you to refrain from putting more than six authors
% (two rows with three columns) beneath the article title.
% More than six makes the first-page appear very cluttered indeed.
%
% Use the \alignauthor commands to handle the names
% and affiliations for an 'aesthetic maximum' of six authors.
% Add names, affiliations, addresses for
% the seventh etc. author(s) as the argument for the
% \additionalauthors command.
% These 'additional authors' will be output/set for you
% without further effort on your part as the last section in
% the body of your article BEFORE References or any Appendices.

\numberofauthors{3} %  in this sample file, there are a *total*
% of EIGHT authors. SIX appear on the 'first-page' (for formatting
% reasons) and the remaining two appear in the \additionalauthors section.

\author{
% You can go ahead and credit any number of authors here,
% e.g. one 'row of three' or two rows (consisting of one row of three
% and a second row of one, two or three).
%
% The command \alignauthor (no curly braces needed) should
% precede each author name, affiliation/snail-mail address and
% e-mail address. Additionally, tag each line of
% affiliation/address with \affaddr, and tag the
% e-mail address with \email.
%
% 1st. author
\alignauthor
Behrooz Omidvar-Tehrani\\
       \affaddr{University of Grenoble Alpes (France)}\\
       \email{behrooz.omidvar-tehrani@univ-grenoble-alpes.fr}
% 2nd. author
\alignauthor
Pl\'acido A. Souza Neto\\
       \affaddr{Federal Institute of Rio Grande do Norte (Brazil)}\\
       \email{placido.neto@ifrn.edu.br}
% 3rd. author
\alignauthor
Tiago Oliveira Lisboa\\
       \affaddr{Federal Institute of Rio Grande do Norte (Brazil)}\\
       \email{tiago.oliveira@\\academico.ifrn.edu.br}
\and
% 4th. author
\alignauthor
Francisco B. Silva Junior\\
       \affaddr{Federal Institute of Rio Grande do Norte (Brazil)}\\
       \email{bento.francisco@\\academico.ifrn.edu.br}
% 5th. author
\alignauthor
Felipe F. Pontes\\
       \affaddr{Federal Institute of Rio Grande do Norte (Brazil)}\\
       \email{freire.pontes@\\academico.ifrn.edu.br}
}

\maketitle


\begin{abstract}
In this paper we present a solution to capture region preferences from implicit feedbacks. Given a point from a spatial dataset, a set of regions captured from gaze and mouse tracking, the set of region intersections, the \sgg\ captures the implicit feedback of analysts and exploits it to highlight potentially interesting options. The analysis consider the concept of relevance and diversity of a given point with the others in the same data set, from the regions tracked during the analysis.  To tackle this challenge, we extend \cite{omidvar2017geoguide} approach by using ST-DBSCAN\cite{Birant:2007} algorithm to map the region preferences from implicit tracking over time. We capture, analyse, generate and save region preferences in order to highlight informations, from any spatial dataset that can be useful to the analyst. So, this work aims to answers questions about interactivity and guidance in spatial solutions. We evaluate the efficiency of the proposed approach experimentally considering spatial datasets.
\end{abstract}

\section{Introduction}
Nowadays, there has been a meteoric rise in the generation of spatial datasets in various fields of science, such as transportation, lodging services and social science. As each record in spatial data represents an activity in a precise geographical location, analyzing such data enables discoveries grounded on facts. Analysts are often interested to observe spatial patterns and trends to improve their decision making process. There exist many applications which require the analysis of spatial data such as smart city management, disaster management and autonomous transport \cite{RoddickEHPS04,Telang:2012}.

\vspace{2pt}
Typically, spatial data analysis begins with an imprecise question in the mind of the analyst, i.e., {\em exploratory analysis}. The analyst requires to go through several trail-and-error iterations to improve her understanding of the spatial data and gain insights. Each iteration involves visualizing a subset of data on geographical maps using an  off-the-shelf product (e.g., Tableau\footnote{\it http://www.tableau.com}, Exhibit\footnote{\it http://www.simile-widgets.org/exhibit/}, Spotfire\footnote{\it http://spotfire.tibco.com}) where the analyst can investigate on different parts of the visualization by zooming in/out and panning the map. 

\vspace{3pt}
The focus in the literature of spatial data analysis is on ``efficiency'', i.e., enabling fluid means of navigation in spatial data. The common approach is to design pre-computed indexes which enable efficient retrieval of spatial data (e.g., \cite{lins2013nanocubes}). However, there has been fewer attention to the ``wisdom'' derived from spatial data. Despite the huge progress on the efficiency front, an analyst may easily get lost in the plethora of geographical points because, first, she doesn't know what to investigate next in an exploratory context and, second, she may get distracted and miss interesting points by visual clutter caused by huge point overlaps.

\vspace{2pt}
The main drawback of the traditional analysis model is that the analyst has a {\em passive role} in the process. In other words, the analyst's feedback (i.e., her likes and dislikes) is ignored and only the input query (i.e., her explicit request) is served. In case feedback is incorporated, the process can be more directed towards analyst's interests where her partial needs can be served earlier in the process. In this paper, we advocate for a ``guidance layer'' on top of the raw visualization of spatial data to enable analysts know {\em ``what to see next''}. This guidance should be a function of analyst feedback: the system should recommend options similar to what the analyst has already appreciated. 
% Hence, feedback capturing lies at the core of such guidance system.

\vspace{2pt}
Various approaches in the literature propose methodologies to incorporate analyst's feedback in the exploration process of spatial data. Typically feedback is considered as a function which is triggered by any analyst's action on the map. The action can ``be selecting a point'', ``moving to a region'', ``asking for more details'', etc. The function then updates a ``profile vector'' which keeps tracks of analyst's interests. The updated content in the profile vector enables the guidance functionality. For instance, if the analyst shows interest in a house with balcony, this choice of amenity will reflect her profile to prioritize such houses in future iterations.

\vspace{2pt}
Feedback is often expressed {\em explicitly}, i.e., the analyst clicks on a point and mentions if she likes or dislikes the point \cite{kamat2014distributed,Omidvar-Tehrani:2015,omidvar2017geoguide}. In \cite{omidvar2017geoguide}, we proposed an interactive approach to exploit such feedback for enabling a more insightful exploration of spatial data. However, there are several cases that the feedback is expressed {\em implicitly}, i.e., the analyst does not explicitly click on a point, but there exists correlations with other signals captured from the analyst which provide hint on her interest. For instance, it is often the case in spatial data analysis that analysts look at some regions of interest but do not provide an explicit feedback. Another example is frequent mouse moves around a region which show the potential interest of the analyst in points in that region. Implicit feedbacks are more challenging to capture and hence less investigated in the literature. The following examples describes a use case of implicit feedbacks in practice.

\vspace{2pt}
\noindent {\bf Running Example.} {\em Benicio is planning to live in Paris for a season. He decides to rent a home-stay from Airbnb website\footnote{\it http://www.airbnb.com}. He likes to discover the city, hence he is open to any type of lodging in any region with an interest to stay in the center of Paris. The system returns 1500 different locations. As he has no other preferences, an exhaustive investigation needs scanning each location independently which is nearly infeasible. While he is scanning few first options, he shows interest in the region of Trocadero (where the Eiffel tower is located in) but he forgets or doesn't feel necessary to click a point there. An ideal system should capture this implicit feedback in order to short-list a small subset of locations that Benicio should consider as high priority}.

\vspace{2pt}
Implicit feedback capturing is crucial in the context of spatial data analysis. While text-boxes, combo-boxes and other elements are available in analyzing other types of data, the only interaction means between the analyst and a spatial data analysis system is a geographical map spanned on the whole screen. A point can be easily remained out of sight on a geographical map.

\vspace{2pt}
In this paper, we present an approach called {\sc GeoPoly} whose aim is to capture and analyze implicit feedback of analysts in spatial data analysis. Without loss of generality, we focus on ``mouse moves'' as the implicit feedback received from the analyst. Mouse moves are the most common way that analysts interact with geographical maps. It is shown in \cite{Arapakis:2014} that mouse gestures have a strong correlation with ``user engagement''. Intuitively, a point gets a higher weight in the analyst's profile if the mouse cursor moves around it frequently.  However, our approach can be easily extended to other types of inputs such gaze tracking, leap motions, etc.

% \cite{Robertson2007}  affirms that temporal change in spatial patterns are increasingly common in geographical analysis. This work explore an approach to the spatialtemporal analysis of polygons that are spatially distinct and experience discrete changes though time. It presents challenges considering changes of regions (polygons) during the time. Works like \cite{Ester:1996} and  \cite{Birant:2007} present solutions for clustering spatialtemporal data. These solutions are relevant to define regions by each cluster that contains important informations for the user. 

% \vspace{3pt}
% Discovering patterns and provide tendencies in spatial data applications may improve insights for planning and decision making for smart city solutions. Many systems and datasets consider space information.  In this way, find spatial preferences can offer interactive and guidance solutions.  For example,  when users look for a house or hotel to spend a season, they consider one or more regions of their preference. These regions are intrinsic to each user, or user group. However, when navigating the application, the user also considers regions that seem interesting, for different reasons, such as the priority of some tourist spot, restaurants, clubs, security, etc. Thus, capturing region preferences over time can help to guide the user to find better places.
 
% \vspace{3pt}
% Given a dataset of spatial points and from the mouse tracking movements by the user, our approach generates a set of highlighted regions based on its preferences. Each region is related with a subset of highlighted points which are illustrated using visual variables such as size and color intensity. The regions are also highlighted.

\vspace{2pt}
The outline of the paper is as follows: Section \ref{sec:datamodel} describes our data model. In Section \ref{sec:overpolygons}, formally define our problem. Then in Section X, we present our solution and its algorithmic details. Section  \ref{sec:experiments} describes our experiments. We conclude in Section \ref{sec:conclusions}.

\section{Data Model}
\label{sec:datamodel}
We consider two different layers on a geographical map: {\em spatial} layer and {\em interaction} layer. The spatial layer contains points from a spatial database $\mathcal{P}$. The interaction layer contains mouse move points $\mathcal{M}$.

\vspace{2pt}
\noindent {\bf Spatial layer.} Each point $p \in \mathcal{P}$ is described using its coordinates, {\em latitude} and {\em longitude}, i.e., $p = \langle \mathit{lat}, \mathit{lon} \rangle$. Note that in this work, we don't consider ``time'' for spatial points as our contribution focuses on their location. Points are also associated to a set of domain-specific attributes $\mathcal{A}$. For instance, for a dataset of a real estate agency, points are properties and $\mathcal{A}$ contains attributes such as ``surface'', ``number of pieces'' and  ``price''. The set of all possible values for an attribute $a \in \mathcal{A}$ is denoted as $dom(a)$. We also define analyst's feedback $F$ as a vector over all attribute values, i.e., $F = \overrightarrow{\cup_{a \in \mathcal{A}}dom(a)}$. The vector $F$ is initialized by zeros and will be manipulated to express analyst's preferences.

\vspace{2pt}
\noindent {\bf Interaction layer} Whenever the analyst moves her mouse, a new point $m$ is appended to the set $\mathcal{M}$. Each mouse move point is described using the pixel position that it touches and the clock time of the move. Hence each mouse move point is a tuple $m = \langle x, y, t \rangle$, where $x$ and $y$ specifies the pixel location and $t$ is a Unix Epoch time. To conform with geographical standards, we assume $m = \langle 0, 0\rangle$ sits at the middle of the interaction layer, both horizontally and vertically. We also define a set of regions $\mathit{R}$ where each $r \in \mathit{R}$ contains a set of mouse move points $\mathcal{M}_r \subseteq \mathcal{M}$ and a time interval $T_r$ where $\forall m=\langle x,y,t\rangle \in \mathcal{M}_r, t \in T_r$.

\vspace{2pt}
The analyst is in contact with the interaction layer. To update the feedback vector $F$, we need to translate pixel locations in the interaction layer to latitudes and longitudes in the spatial layer. While there is no precise transformation from planar to spherical coordinates, we employ equirectangular projection to obtain the best projection. Equation \ref{eq:equirectangular} describes this formula to transform a point $m = \langle x,y,t \rangle$ in the interaction layer to a point $p = \langle lat, lon \rangle$ in the spatial layer. Note that the resulting $p$ is not necessarily a member of $\mathcal{P}$. 

\begin{equation}\label{eq:equirectangular}
\mathit{lon} = \frac{x}{\mathit{cos}\gamma} + \theta; \mathit{lat} = y + \gamma 
\end{equation}

The inverse operation, i.e., transforming from the spatial layer to the interaction is done using Equation \ref{eq:reverse}.

\begin{equation}\label{eq:reverse}
x = (\mathit{lon} - \theta) \times \mathit{cos}\gamma; y = \mathit{lat} - \gamma
\end{equation}

\vspace{2pt}
The reference point for the transformation is the center of both layers. In Equation \ref{eq:equirectangular}, we assume that $\gamma$ and $\theta$ are the respective latitude and longitude of a point in the spatial layer which sits at the center of the interaction layer, i.e., $m= \langle 0,0 \rangle$.

\section{Problem Definition}
We tackle the challenge of ``information overload'' in spatial data by addressing the problem of information highlighting, i.e., highlight few points based on analyst's interests in order to guide analysts towards what they should focus on in consecutive iterations of the analysis process. We formally define our problem as follows.

\vspace{2pt}
\noindent {\bf Problem.} {\em Given a time $t_c$ and an integer constant $k$, obtain an updated feedback vector $F$ using points $m \in \mathcal{M}$ where $m.t \leq t_c$ and choose $k$ points $\mathcal{P}_k \subseteq \mathcal{P}$ as highlights where $\mathcal{P}_k$ satisfies two following constraints.}

\vspace{2pt}
\noindent - $\forall p \in \mathcal{P}_k, \mathit{similarity}(p,F)$ {\em is maximized.}

\vspace{2pt}
\noindent - $\mathit{diversity}(\mathcal{P}_k)$ {\em is maximized.}

\vspace{2pt}
The first constraint guarantees that returned highlights are highly similar with analyst's interests captured in $F$. The second constraint ensures that $k$ points cover different regions and they don not repeat themselves. While our approach is independent from the way that the functions {\em similarity} and {\em diversity} are formulated, we provide formal definitions of this functions in Section \ref{sec:algo}.

\vspace{2pt}
The aforementioned problem is hard to solve due to following challenges.

\vspace{2pt}
\noindent $\blacksquare$ {\bf Challenge 1.} First it is not clear how mouse move points influence the feedback vector. Mouse moves occur on a separate layer and there should be some meaningful transformations to interpret mouse moves as potential changes in the feedback vector. 

\vspace{2pt}
\noindent $\blacksquare$ {\bf Challenge 2.} Even if an oracle provides a mapping between mouse moves and the feedback vector, analyzing all generated mouse moves is challenging and may introduce false positives. A typical mouse with 1600 DPI (Dots Per Inch), touches 630 pixels for one centimeter of move. Hence a mouse move from the bottom to the top of a typical 13-inch screen would provide 14,427 points which may not be necessarily meaningful.

\vspace{2pt}
\noindent $\blacksquare$ {\bf Challenge 3.} Beyond two first challenges, finding the most similar and diverse points with $F$ needs an exhaustive scan of all points in $\mathcal{P}$ which is prohibitively expensive: in most spatial dataset, there exists millions of points. Moreover, we need to follows multi-objective considerations as we aim to optimize both similarity and diversity at the same time.

\vspace{2pt}
In Section \ref{sec:algo}, we discuss a framework called {\sc GeoPoly} as a solution for the aforementioned problem and its challenges.

\section{GeoPoly Framework}
\label{sec:algo}
{\sc GeoPoly} is an approach which exploits analyst's implicit feedback (i.e., mouse moves) to highlight few interesting points as future analysis directions. Algorithm \ref{algo:main} summarizes the important steps of our approach.

\begin{algorithm}[t]
\DontPrintSemicolon
\KwIn{Current time $t_c$, mouse move points $\mathcal{M}$}
\KwOut{Highlights $\mathcal{P}_k$}
$\mathcal{S} \gets \mathit{find\_interesting\_dense\_regions}(t_c,\mathcal{M})$\label{ln:dense}\;
$\mathcal{P}_s \gets \mathit{match\_points}(\mathcal{S}, \mathcal{P})$\label{ln:match}\;
$F \gets \mathit{update}(F, \mathcal{P}_s)$\label{ln:update}\;
$\mathcal{P}_k \gets \mathit{get\_highlights}(\mathcal{P}, F)$\label{ln:highlight}\\;
\Return{$\mathcal{P}_k$}\; 
\caption{{\sc GeoPoly} Algorithm}
\label{algo:main}
\end{algorithm}

\vspace{2pt}
The algorithm begins by mining the set of mouse move points $\mathcal{M}$ in the interaction layer to discover one or several Interesting Dense Regions, abbr., IDRs, in which most analyst's interactions occur (line \ref{ln:dense}). Then it matches the dense area with the spatial layer using Equation \ref{eq:equirectangular} to find spatial points inside the area (line \ref{ln:match}). The attributes of resulting points will be exploited to update the analyst's feedback vector $F$ (line \ref{ln:update}). The updated vector $F$ will then be used to find $k$ highlights (line \ref{ln:highlight}). These steps ensure that the final highlights reflect analyst's implicit interests. we detail each step as follows.

\subsection{Interesting Dense Regions}
The objective of this step is to obtain one or several regions in which the analyst has expressed her implicit feedback. There are two observations for such regions.

\vspace{2pt}
\noindent $\blacksquare$ {\bf Observation 1.} We believe that a region is more interesting to the analyst if it is denser, i.e., the analyst moves her mouse in that region several times.

\vspace{2pt}
\noindent $\blacksquare$ {\bf Observation 2.} It is possible that the analyst moves her mouse everywhere in the map. This should not signify that everywhere in the map has the same significance.

\vspace{2pt}
Following our observations, we propose Algorithm \ref{algo:dense} for mining IDRs. We add points to $\mathcal{M}$ only every $200ms$ to prevent adding redundant points.  Following Observation 1 and in order to mine the recurring behavior of the analyst, the algorithm begins by partitioning the set $\mathcal{M}$ into $g$ fixed-length consecutive segments $\mathcal{M}_0$ to $\mathcal{M}_g$. The first segment starts at time zero (where the system started), and the last segment ends at $t_c$, i.e., the current time. Following Observation 2, we then find dense clusters in each segment of $\mathcal{M}$ using a variant of DB-SCAN approach. Finally, we return intersections among those clusters as IDRs.

\begin{algorithm}[t]
\DontPrintSemicolon
\KwIn{Current time $t_c$, mouse move points $\mathcal{M}$}
\KwOut{IDRs $\mathcal{S}$}
$\mathcal{S} \gets \emptyset$\;
$g \gets ${\em number of time segments}\;
\For{$i \in [0,g]$}
{
       $\mathcal{M}_i \gets \{m = \langle x,y,t \rangle | (\frac{t_c}{g} \times i) \leq t \leq (\frac{t_c}{g} \times (i+1))\}$\;
       $\mathcal{C}_i \gets \mathit{mine\_clusters}(\mathcal{M}_i)$\label{ln:mine}\;
       $\mathcal{O}_i \gets \mathit{find\_ploygons}(\mathcal{C}_i)$\label{ln:poly}\;
}
\lFor{$\mathcal{O}_i, \mathcal{O}_j$ where $i,j \in [0,g]$ and $i \neq j$}
{
       $\mathcal{S}.\mathit{append}(\mathit{intersect}(\mathcal{O}_i, \mathcal{O}_j))$
}
\Return{$\mathcal{S}$}\; 
\caption{Find Interesting Dense Regions (IDRs)}
\label{algo:dense}
\end{algorithm}

\vspace{2pt}
For clustering points in each time segment (line \ref{ln:mine}), we use ST-DBSCAN, a space-aware variant of DB-SCAN for clustering points based on density \cite{Birant:2007}. For each subset of mouse move points $\mathcal{M}_i$, $i \in [0,g]$, ST-DBSCAN begins with a random point $m_0 \in \mathcal{M}_i$ and collects all density-reachable points from $m_0$ using a distance metric. As mouse move points are in form of pixels in a 2-dimensional space (i.e., the display), we choose euclidean distance as the distance metric. If $m_0$ turns out to be a core object, a cluster will be generated. Otherwise, if $m_0$ is a border object, no points are density-reachable from $m_0$ and the algorithm picks another random point in $\mathcal{M}_i$. The process is repeated until all of the points have been processed.

\vspace{2pt}
Once clusters are obtained for all subsets of $\mathcal{M}$, we find their intersections to locate recurring regions (line \ref{ln:poly}). To obtain intersections, we need to clearly define the spatial boundaries of each cluster. Hence for each cluster, we discover its corresponding polygon that covers the points inside. For this aim, we employ Quickhull algorithm, a quicksort-style method which computes the convex hull for a given set of points in a 2D plane~\cite{Barber:1996}.

\begin{figure*}[t]
\centering
   \includegraphics[width=\textwidth]{imgs/regions}
  \caption{The process of finding IDRs on Airbnb dataset.}
  \label{fig:regions}
\end{figure*}

\vspace{2pt}
We describe the process of finding IDRs in an example. Figure \ref{fig:regions} shows the steps that Benecio in our running example follows. Figure \ref{fig:regions}.A shows mouse movements of Benecio in different time stages. In this example, we consider $g = 3$ and capture Benecio's feedback in three different time segments (progressing from Figures \ref{fig:regions}.B to \ref{fig:regions}.D). It shows that Benecio started his search around Eiffel Tower and Arc de Triomphe (Figure \ref{fig:regions}.B) and gradually showed interest in south (Figure \ref{fig:regions}.C) and north (Figure \ref{fig:regions}.D) as well. All intersections between those clusters are discovered (hatching regions in Figure \ref{fig:regions}.E) which will constitute the set of IDRs (Figure \ref{fig:regions}.F), i.e., IDR1 to IDR4.

\subsection{Matching Points}
Being a function of mouse move points, IDRs are discovered in the interaction layer. We then need to find out which points in $\mathcal{P}$ fall into IDRs. We employ Equation \ref{eq:reverse} to transform those points from the spatial layer to the interaction layer. Then a simple ``spatial containment'' function can verify which points fit into the IDRs. Given a point $p$ and a region $r$, a function $\mathit{contains}(p,r)$ returns ``true'' if $p$ is inside $r$, otherwise ``false''. In our case, we simply use $\mathit{ST\_Within}(p,r)$ module in PostGIS\footnote{\it https://postgis.net/docs/manual-dev/ST_Within.html}.

\begin{figure}[t]
\centering
   \includegraphics[width=\columnwidth]{imgs/match}
  \caption{Matching points for IDR1 to IDR4.}
  \label{fig:match}
\end{figure}

\vspace{2pt}
In the vanilla version of our spatial containment function, all points should be checked against all IDRs. Obviously, this depletes the execution time. To prevent the exhaustive scan, we employ Quadtrees in two different steps.

\vspace{4pt}
\noindent $\blacksquare$ In an offline process, we build a Quadtree index for all points in $\mathcal{P}$. We record the membership relations of points and cells in the index.

\vspace{2pt}
\noindent $\blacksquare$ When IDRs are discovered, we record which cells in the Quadtree index intersect with IDRs. Hence for matching points, we only check a subset which is inside a cell associated to IDRs.

\vspace{4pt}
We follow our running example and illustrate the matching process in Figure \ref{fig:match}. In the Airbnb dataset, points are home-stays which are shown with their nightly price on the map. We observe that there exist many matching points with IDR3 and absolutely no matching point for IDR2. For IDR4, although there exist many home-stays below the region, we never check their containment, as they belong to a Quadtree cell which doesn't intersect with the IDR. 

\subsection{Updating Analyst Feedback Vector}
The set of matching points $\mathcal{P}_s$ depicts the implicit preference of the analyst. We keep track of this preference in a feedback vector $F$. The vector is initialized by zero, i.e., the analyst has no preference at the beginning. We update $F$ using the attributes of the points in $\mathcal{P}_s$.

\vspace{2pt}
We consider an {\em increment value} $\delta$ to update $F$. If $p \in \mathcal{P}_s$ gets $v_1$ for attribute $a_1$, we augment the value in the $F$'s cell of $\langle a_1, v_1 \rangle$ by $\delta$. Note that we only consider incremental feedback.

\begin{table}[]
\centering
\caption{Attributes of points in IDR1.}
\label{tbl:attribs}
\begin{tabular}{|c|c|c|c|c|c|}
\hline
\textbf{ID} & \textbf{Price} & \textbf{\#Beds} & \textbf{Balcony} & \textbf{Air-cond.} & \textbf{Rating} \\ \hline
1                     & 130\euro           & 1               & Yes           & Yes                & 5/5             \\ \hline
2                     & 58\euro            & 1               & Yes           & No                 & 5/5             \\ \hline
3                     & 92\euro            & 2               & Yes           & No                 & 5/5             \\ \hline
4                     & 67\euro            & 1               & Yes           & No                 & 4/5             \\ \hline
\end{tabular}
\end{table}

\begin{table}[]
\centering
\caption{Updating Analyst Feedback Vector}
\label{tbl:feedback}
\begin{tabular}{|c|c|c|}
\hline
\textbf{Attribute-value}               & \textbf{Applying IDR 1} & \textbf{Normalized} \\ \hline
$\langle$\#Beds,1$\rangle$                   & $+3\delta$                       & 0.19                 \\ \hline
$\langle$\#Beds,2$\rangle$                 & $+\delta$                       & 0.06                 \\ \hline
$\langle$\#Beds,+2$\rangle$                  & {\em (no update)}                       & 0.00                    \\ \hline
$\langle$Balcony,Yes$\rangle$                   & $+4\delta$                      & 0.25                 \\ \hline
$\langle$Balcony,No$\rangle$                    & {\em (no update)}                        & 0.00                    \\ \hline
$\langle$Air-cond.,Yes$\rangle$               & $+\delta$                       & 0.06                 \\ \hline
$\langle$Air-cond.,No$\rangle$                & $+3\delta$                       & 0.19                 \\ \hline
$\langle$Rating,1$\rangle$                    & {\em (no update)}                       & 0.00                    \\ \hline
$\langle$Rating,2$\rangle$                     & {\em (no update)}                        & 0.00                    \\ \hline
$\langle$Rating,3$\rangle$                    & {\em (no update)}                        & 0.00                   \\ \hline
$\langle$Rating,4$\rangle$                   & $+\delta$                       & 0.06                 \\ \hline
$\langle$Rating,5$\rangle$                     & $+3\delta$                      & 0.19                 \\ \hline
\end{tabular}
\end{table}

\vspace{2pt}
We explain the process of updating the feedback vector by a toy example. Given the four matched points in IDR1 (Figure \ref{fig:match}) with prices 130\euro, 58\euro, 92\euro\ and 67\euro, we want update the vector $F$ given those points. Few attributes of these points are mentioned in Table \ref{tbl:attribs}. In practice, there are typically more than 50 attributes for points. The cells of $F$ is illustrated in the first column of Table \ref{tbl:feedback}. As three points get ``1'' for the attribute \#Beds, then the value in cell $\langle$\#Beds,1$\rangle$ is augmented three times by $\delta$. The same process is repeated for all attribute-values of points in $\mathcal{P}_s$. Note that all cells of $F$ are not necessarily touched in an update process. For instance, in the above example, 5 cells out of 12 remain unchanged.

\vspace{2pt}
By specifying an increment value, we can materialize the updates and normalize the vector using a Softmax function. We always normalize $F$ in a way that all cell values sum up to $1.0$. Given $\delta = 1.0$, the normalized values of the $F$ vector is illustrated in the third column of Table \ref{tbl:feedback}. Higher values of $\delta$ increase the influence of feedbacks.

\vspace{2pt}
The normalized content of the vector $F$ captures the implicit preferences of the analyst. For instance, the content of $F$ after applying points in IDR1 shows that the analyst has a high interest in having a balcony in her home-stay, as her score for the cell $\langle$Balcony,Yes$\rangle$ is 0.25, i.e., the highest among other cells. This reflects the reality as all points in IDR1 has balcony. Note that although we only consider positive feedback, the Softmax function lowers the values of untouched cells once other cells get rewarded.

\vspace{2pt}
An important consideration in interpreting the vector $F$ is that the value ``0'' does not mean the lowest preference, but {\em irrelevance}. For instance, consider the cell $\langle$Rating,2$\rangle$ in Table \ref{tbl:feedback}. The value ``0'' for this cell shows that the analyst has never expressed her implicit feedback on this aspect. It is possible that in future iterations, the analyst shows interest in a 2-star home-stay (potentially thanks to its price), hence this cell gets a value greater than zero. However, cells with lower preferences are identifiable with non-zero values tending to zero. For instance, the value 0.06 for the cell $\langle$Rating,4$\rangle$ shows a lower preference towards 4-star home-stays comparing to the ones with 5 stars, as only one point is rated 4 in IDR1.

% SORT
\subsection{Generating Highlights}
The ultimate goal of \sgg\ is to highlight $k$ points to guide analysts in analyzing their spatial data. The updated feedback vector $F$ is the input to the highlighting phase. We assume that points in IDRs are already investigated by the analyst. Hence our search space contains all points in $\mathcal{P}$ except ones inside IDRs.

\vspace{2pt}
We seek two properties in $k$ highlights, i.e., {\em similarity} and {\em diversity}. First, highlights should be in the same direction of the analyst's implicit feedback, hence similar to the vector~$F$. The similarity between a point $p \in \mathcal{P}$ and the vector~$F$ is defined as follows.

\begin{equation}
       \label{eq:rel}
       \mathit{similarity}(p,F) = \mathit{avg}_{a \in \mathcal{A}}(\mathit{sim(p, F, a)})
\end{equation}

The $\mathit{sim}()$ function can be any function such as Jaccard and Cosine. Each attribute can have its own similarity function (as string and integer attributes are compared differently.) Then $\mathit{sim}()$ works as an overriding-function which provides encapsulated similarity computations for any type of attribute.

\vspace{2pt}
Second, highlighted points should also represent distinct directions so that the analyst can observe different aspects of data and decide based on the big picture. Given a set of points $\mathcal{P}_k = \{ p_1, p_2 \dots p_k \} \subseteq {\cal P}$, we define {\em diversity} as follows.

\begin{equation}
       \label{eq:divs}
       \mathit{diversity}(\mathcal{P}_k) = \mathit{avg}_{\{p, p'\} \subset \mathcal{P}_k | p \neq p' } \mathit{distance}(p,p')
\end{equation} 

The function $\mathit{distance}(p,p')$ operates on geographical coordinates of $p$ and $p'$ and can be considered as any distance function of Minkowski distance family. However, as distance computations are done in the spherical space, a natural choice is to employ Haversine distance shown in Equation~\ref{eq:harvestine}.

\begin{dmath}
       \label{eq:harvestine}
       distance(p,p') = acos(cos(p.\mathit{lat}) \times cos(p'.\mathit{lat}) \times cos(p.\mathit{lon})) \times cos(p'.\mathit{lon}) + cos(p.\mathit{lat}) \times sin(p'.\mathit{lat}) \times cos(p.\mathit{lon}) \times sin(p'.\mathit{lon}) + sin(p.\mathit{lat}) \times sin(p'.\mathit{lat})) \times earth\_radius
\end{dmath}

Algorithm \ref{algo:geoh} describes our approach for highlighting $k$ similar and diverse points.
We propose a best-effort greedy approach to efficiently compute highlighted points. We consider an offline step followed by the online execution of our algorithm.

\vspace{2pt}
In order to speed up the similarity computation in the online execution, we pre-compute an inverted index for each single point $p \in {\cal P}$ in the offline step (as is commonly done in Web search). Each index ${\cal L}_p$ for the point $p$ keeps all other points in ${\cal P}$ in decreasing order of their relevance with $p$.

\vspace{2pt}
The first step of Algorithm \ref{algo:geoh} is to find the most similar point to $F$, so-called $p^*$. The point $p^*$ is the closest possible approximation of $F$ in order to exploit pre-computed similarities. The algorithm makes sequential accesses to ${\cal L}_{p^*}$ (i.e., the inverted index of the point $p^*$) to greedily maximize diversity. Algorithm \ref{algo:geoh} does not sacrifice efficiency in price of value. We consider a {\em time limit} parameter which determines when the algorithm should stop seeking maximized diversity. Scanning inverted indexes guarantees the similarity maximization even if time limit is chosen to be very restrictive. Our observations with several spatial datasets show that we achieve the diversity of more than $0.9$ with time limit set to $200ms$.

% %\noindent{\bf Context.} 

\begin{algorithm}[t]
\DontPrintSemicolon
\KwIn{Points $\mathcal{P}$, Feedback vector $F$, $k$, $\mathit{time\_limit}$}
\KwOut{$\mathcal{P}_k$}
$p^* \gets \mathit{max\_sim\_to}(\mathcal{P},F)$\;
$\mathcal{P}_k \gets \mathit{top\_k}(\mathit{{\cal L}_{p^*}},k)$\label{ln:topk}\;
$p_{next} \gets get\_next(\mathit{{\cal L}_{p^*}})$\;\label{cd:getnext}
\While{$\mathit{time\_limit}$ $not$ $exceeded$}
       {\label{cd:beginwhile}
       \For{$p_{current} \in {\cal P}_k$}
              {
              \If{$\mathit{diversity\_improved}({\cal P}_k,p_{next},p_{current})$}
                     {\label{cd:betterdiv}
                     ${\cal P}_k \gets \mathit{replace}({\cal P}_k,p_{next},p_{current})$\;
                            $break$\;
                     }
              }
              $p_{next} \gets get\_next(\mathit{{\cal L}_{p^*}})$\;}\label{cd:endwhile}
       \Return{${\cal P}_k$}\; 
       \caption{Get $k$ similar and diverse highlights $\mathit{get\_highlights}()$}
       \label{algo:geoh}
\end{algorithm}

\vspace{2pt}
In line \ref{ln:topk} of Algorithm \ref{algo:geoh}, $\mathcal{P}_k$ is initialized with the $k$ highest ranking points in ${\cal L}_{p^*}$. Function $get\_next({\cal L}_{p^*})$ (line \ref{cd:getnext}) returns the next point $p_{next}$ in ${\cal L}_{p^*}$ in sequential order. Lines \ref{cd:beginwhile} to \ref{cd:endwhile} iterate over the inverted indexes to determine if other points should be considered to increase diversity while staying within the time limit and not violating the relevance threshold with the selected point.

\vspace{2pt}
The algorithm looks for a candidate point $p_{\mathit current} \in {\cal P}_k$ to replace in order to increase diversity. The boolean function $\mathit{diversity\_improved}()$ (line \ref{cd:betterdiv}) checks if by replacing $p_{current}$ by $p_{next}$ in ${\cal P}_k$, the overall diversity of the new ${\cal P}_k$ increases.



% Algorithm \ref{algo:geoh} modifies the original {\sc Highlighter} proposal presented in GeoGuide \cite{Omidvar:2017} approach .  
% The original begins by retrieving the most relevant points to $p$ by simply retrieving the $k$ highest ranking points in ${\cal L}_p$ (line \ref{cd:gettopk}) and function $get\_next({\cal L}_p)$ (Line \ref{cd:getnext}) returns the next point $p_{next}$ in ${\cal L}_p$ in sequential order. Line \ref{cd:empty_regions} initialize the set of points that will be retrieved by the highlighted regions. At the beginning we consider that there is no preferred regions. So, the sets ${\cal S}_{rp}$ and ${\cal S}_{p}$  are the same. Lines \ref{cd:beginwhile} to \ref{cd:endwhile} iterate over the inverted indexes to determine if other points should be considered to increase diversity while staying within the time limit and not violating the relevance threshold with the selected point. %Since points in ${\cal L}_g$ are sorted on decreasing relevance with $p$, the algorithm can safely stop as soon as the relevance condition is violated (or if the time limit is exceeded).

% The algorithm then looks for a candidate point $p_{current} \in {\cal S}_p$ to replace in order to increase diversity. If the candidate point is presented in a region $r$ (line \ref{cd:point_in_region}), the point is included in ${\cal S}_{rp}$, considering the boolean function $\mathit{diversity\_improved}()$ (line \ref{cd:betterdiv}). This function checks if by replacing $p_{current}$ by $p_{next}$ in ${\cal S}_p$, the overall diversity of the new ${\cal S}_p$ increases. If the point is not in the preferred region, it is included into ${\cal S}_p$.

\section{Experiments}
\label{sec:experiments}
In this section, we show some quantitative and qualitative experiments.

\section{Related Work}
The literature contains several instances of feedback exploitation to guide the analyst in further analysis steps (e.g., \cite{boley2013one}). The common approach is a top-$k$ processing methodology in order to prune the search space based on the explicit feedback and recommend a small subset of interesting results of size~$k$. A clear distinction of \sgg\ is that it doesn't aim for pruning, but leveraging the actual data with potential interesting results that the analyst may miss due to the huge volume of spatial data. While in top-$k$ processing algorithms, analyst choices are limited to $k$, \sgg\ has a freedom of choice where highlights get seamlessly updated with new analyst choices. 

\vspace{2pt}
There exist few instances of information-highlighting methods in the literature \cite{Liang2010,Robinson2011,wongsuphasawat2016voyager,willett2007scented}. All these methods are {\em objective} and do not apply to the context of spatial guidance where user feedback is involved.  In terms of recommendation, few approaches focus on spatial dimension \cite{Bao2015,Levandoski:2012} while the context and result diversification are missing.

\section{Conclusion}
\label{sec:conclusions}
Here we conclude.

\bibliographystyle{abbrv}
\bibliography{main}

\end{document}
