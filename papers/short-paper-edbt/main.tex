\documentclass{sig-alternate-05-2015}
\usepackage{algorithm} 
\usepackage{algorithmic}
\usepackage{breqn}

 
\begin{document}

\setcopyright{acmcopyright}

%\setcopyright{acmlicensed}
%\setcopyright{rightsretained}
%\setcopyright{usgov}
%\setcopyright{usgovmixed}
%\setcopyright{cagov}
%\setcopyright{cagovmixed}

\newtheorem{problem}{Problem}
\newtheorem{definition}{Definition}

 
% DOI
\doi{10.475/123_4}

% ISBN
\isbn{123-4567-24-567/08/06}

%Conference
\conferenceinfo{PLDI '13}{June 16--19, 2013, Seattle, WA, USA}

\acmPrice{\$15.00}

\conferenceinfo{EDBT}{2016}

\title{GeoHighlight: A Point-Recommendation Approach for Spatiotemporal Data}
%\subtitle{[Extended Abstract]

\numberofauthors{3}


\author{
Behrooz Omidvar-Tehrani$^{\dag}$, Gustavo Guerino$^{\ddag}$, Pl\'acido A. Souza Neto$^{\ddag}$\\
\affaddr{
$^{\dag}$The Ohio State University, USA, $^{\ddag}$Federal Institute of Rio Grande do Norte - IFRN, Brazil}\\
\affaddr{
$^{\dag}$\path{omidvar-tehrani.1@osu.edu},
$^{\ddag}$\path{gustavo.guerino@academico.ifrn.edu.br},
$^{\ddag}$\path{placido.neto@ifrn.edu.br}
}}

% \date{30 July 1999}
% Just remember to make sure that the TOTAL number of authors
% is the number that will appear on the first page PLUS the
% number that will appear in the \additionalauthors section.

\maketitle
\begin{abstract}

This paper proposes an approach towards generic visualization of
spatiotemporal data. Although the initial focus of our proposal is on urban
data, our framework was designed in a way to support any types of
\textit{geo} data such as flight data, bike data, smartphone GPS data, etc.
Highlighting spatiotemporal data is also a challenge that this work exploits, by using
recommendation algorithm to produce relevant results from datasets that has
geographic informations. We also discuss some directions and benefits of using
recommendation algorithm to highlight or retrieve important informations from
these kind of datasets.

\end{abstract}

\keywords{Interactive analysis; Spatiotemporal visualization; Urban data.}

\newpage
\section{Introduction}
Nowadays, there has been a meteoric rise in the generation of spatial datasets in various fields of science, such as transportation, lodging services and social science. As each record in spatial data represents an activity in a precise geographical location, analyzing such data enables discoveries grounded on facts. Analysts are often interested to observe spatial patterns and trends to improve their decision making process. Spatial data analysis has various applications such as smart city management, disaster management and autonomous transport \cite{RoddickEHPS04,Telang:2012}.

\vspace{2pt}
Typically, spatial data analysis begins with an imprecise question in the mind of the analyst, i.e., {\em exploratory analysis}. The analyst requires to go through several trail-and-error iterations to improve her understanding of the spatial data and gain insights. Each iteration involves visualizing a subset of data on geographical maps using an  off-the-shelf product (e.g., Tableau\footnote{\it http://www.tableau.com}, Exhibit\footnote{\it http://www.simile-widgets.org/exhibit/}, Spotfire\footnote{\it http://spotfire.tibco.com}) where the analyst can investigate on different parts of the visualization by zooming in/out and panning on the map. 

\vspace{3pt}
Spatial data are often voluminous. Hence the focus in the literature of spatial data analysis is on ``efficiency'', i.e., enabling fluid means of navigation in spatial data to facilitate the exploratory analysis. The common approach is to design pre-computed indexes which enable efficient retrieval of spatial data (e.g., \cite{lins2013nanocubes}). However, there has been fewer attention to the ``value'' derived from spatial data. Despite the huge progress on the efficiency front, an analyst may easily get lost in the plethora of geographical points due to two following reasons.

\vspace{3pt}
\noindent $\blacksquare$ In an exploratory context, the analyst doesn't know apriori what to investigate next.

\vspace{2pt}
\noindent $\blacksquare$ Moreover, she may easily get distracted and miss interesting points by visual clutter caused by huge point overlaps.

\vspace{3pt}
The main drawback of the traditional analysis model is that the analyst has a {\em passive role} in the process. In other words, the analyst's feedback (i.e., her likes and dislikes) is ignored and only the input query (i.e., her explicit request) is served. In case feedback is incorporated, the process can be more directed towards analyst's interests where her partial needs can be served earlier in the process. In this paper, we advocate for a ``guidance layer'' on top of the raw visualization of spatial data to enable analysts know {\em ``what to see next''}. This guidance should be a function of analyst feedback: the system should recommend options similar to what the analyst has already appreciated. 
% Hence, feedback capturing lies at the core of such guidance system.

\vspace{2pt}
Various approaches in the literature propose methodologies to incorporate analyst's feedback in the exploration process of spatial data. Typically, feedback is considered as a function which is triggered by any analyst's action on the map. The action can be ``selecting a point'', ``moving to a region'', ``asking for more details'', etc. The function then updates a ``profile vector'' which keeps tracks of analyst's interests. The updated content in the profile vector enables the guidance functionality. For instance, if the analyst shows interest in a point which describes a house with balcony, this choice of amenity will reflect her profile to prioritize other houses with balcony in future iterations.

\vspace{2pt}
Feedback is often expressed {\em explicitly}, i.e., the analyst clicks on a point and mentions if she likes or dislikes the point \cite{kamat2014distributed,Omidvar-Tehrani:2015,omidvar2017geoguide}. In \cite{omidvar2017geoguide}, we proposed an interactive approach to exploit such feedback for enabling a more insightful exploration of spatial data. However, there are several cases that the feedback is expressed {\em implicitly}, i.e., the analyst does not explicitly click on a point, but there exists correlations with other signals captured from the analyst which provide hint on her interest. For instance, it is often the case in spatial data analysis that analysts look at some regions of interest but do not provide an explicit feedback. Another example is frequent mouse moves around a region which is a good indicator of the analyst's potential interest in the points in that region. Implicit feedbacks are more challenging to capture and hence less investigated in the literature. The following example describes a use case of implicit feedbacks. This will be our running example which we follow thoughout the paper.

\vspace{2pt}
\noindent {\bf Example.} {\em Ben\'icio is planning to live in Paris for a season. He decides to rent a home-stay from Airbnb website\footnote{\it http://www.airbnb.com}. He likes to discover the city, hence he is open to any type of lodging in any region with an interest to stay in the center of Paris. The website returns 1500 different locations. As he has no other preferences, an exhaustive investigation needs scanning each location independently which is nearly infeasible. While he is scanning few first options, he shows interest in the region of Trocadero (where the Eiffel tower is located in) but he forgets or doesn't feel necessary to click a point there. An ideal system should capture this implicit feedback in order to short-list a small subset of locations that Ben\'icio should consider as high priority}.

\vspace{2pt}
The above example shows in practice that implicit feedback capturing is crucial in the context of spatial data analysis. While text-boxes, combo-boxes and other input elements are available in analyzing other types of data, the only interaction means between the analyst and a spatial data analysis system is a geographical map spanned on the whole screen. In this context, a point can be easily remained out of sight and missed.

\vspace{2pt}
In this paper, we present an approach called {\sc GeoPoly} whose aim is to capture and analyze implicit feedback of analysts in spatial data analysis. Without loss of generality, we focus on ``mouse moves'' as the implicit feedback received from the analyst. Mouse moves are the most common way that analysts interact with geographical maps~\cite{Chen:2001}. It is shown in \cite{Arapakis:2014} that mouse gestures have a strong correlation with ``user engagement''. Intuitively, a point gets a higher weight in the analyst's profile if the mouse cursor moves around it frequently.  However, our approach can be easily extended to other types of inputs such gaze tracking, leap motions, etc.

% \cite{Robertson2007}  affirms that temporal change in spatial patterns are increasingly common in geographical analysis. This work explore an approach to the spatialtemporal analysis of polygons that are spatially distinct and experience discrete changes though time. It presents challenges considering changes of regions (polygons) during the time. Works like \cite{Ester:1996} and  \cite{Birant:2007} present solutions for clustering spatialtemporal data. These solutions are relevant to define regions by each cluster that contains important informations for the user. 

% \vspace{3pt}
% Discovering patterns and provide tendencies in spatial data applications may improve insights for planning and decision making for smart city solutions. Many systems and datasets consider space information.  In this way, find spatial preferences can offer interactive and guidance solutions.  For example,  when users look for a house or hotel to spend a season, they consider one or more regions of their preference. These regions are intrinsic to each user, or user group. However, when navigating the application, the user also considers regions that seem interesting, for different reasons, such as the priority of some tourist spot, restaurants, clubs, security, etc. Thus, capturing region preferences over time can help to guide the user to find better places.
 
% \vspace{3pt}
% Given a dataset of spatial points and from the mouse tracking movements by the user, our approach generates a set of highlighted regions based on its preferences. Each region is related with a subset of highlighted points which are illustrated using visual variables such as size and color intensity. The regions are also highlighted.

\vspace{2pt}
The outline of the paper is the following. Section \ref{sec:datamodel} describes our data model. In Section \ref{sec:problem}, we formally define our problem. Then in Section \ref{sec:algo}, we present our solution and its algorithmic details. Section  \ref{sec:exp} reports our experiments on the framework. We review the related work in Section \ref{sec:rel}. Last, we conclude in Section \ref{sec:conc}.
\section{Data Model}\label{sec:data-model}
% prune this
A wide range of spatiotemporal data is present in a vast variety of datasets such as aviation, ground transportation (bike, taxi, renting- car, bus), urban data, geo-tagged social networks, crimes, events, etc. Intuitively, the common point between all those dataset is having {\em location} and {\em time} attributes. Based on this particularity, we propose a generic data model to capture all diverse aspects of such data.

We consider a spatiotemporal database ${\cal D}$ consisting $\langle {\cal P}, {\cal A} \rangle$ where ${\cal P}$ is the set of
geographical points and ${\cal A}$ is the set of point attributes. For each $p \in {\cal P}$, we consider a tuple $<id, lat, lon, alt, t>$ where $id$ is the point identifier, $lat$, $lon$ and $alt$ denote $p$'s geographical coordinates (latitude, longitude and altitude respectively), and $t$ is the timestamp.

The set ${\cal A}_p$ contains attribute-values for $p$ over the schema of ${\cal A}$. For instance, on a bike-sharing dataset, ${\cal A}_p = \langle $ {\tt female}, {\tt young}, {\tt subscribed} $\rangle$ on the schema ${\cal A} = \langle$ {\tt gender}, {\tt age-category}, {\tt subscription} $\rangle$ denotes that $p$ is associated to a young female cyclist who is subscribed in the bike-sharing system. The set ${\cal A}$ is domain-dependent and defines the semantics of a spatiotemporal dataset. For instance, in case of a taxi dataset, ${\cal A} = \langle$ {\tt dropoff\_time}, {\tt price}, {\tt tip} $\rangle$, where for an aviation dataset, ${\cal A} = \langle$ {\tt aircraft\_type}, {\tt departure\_airport}, {\tt arrival\_airport} $\rangle$.

% prune this
Some spatiotemporal datasets contain point-sets as entities, such as {\em trajectories} in transportation datasets and {\em regions} in urban or agriculture dataset. Although our generic data model only captures the finest granular concept (i.e., point), we define ${\cal S}$ containing point-sets. Each point-set $s \in {\cal S}$ is indeed a set of points where $s \subseteq {\cal P}$. For instance, in a taxi dataset, $s = [ p_1, p_2 \dots p_n ]$ shows a ride consisting $n$ points departing at $p_1$ and arriving at $p_n$.
\section{Problem Statement}
\label{sec:pb}
% In an exploratory analysis context, the analyst does not necessarily know what to ask. She may have also a few knowledge about the spatiotemporal data and its attributes. Hence she usually needs to take iterative analysis steps to observe different aspects of data and ultimately land on a subset of interest. However, it is often cumbersome to choose what to analyze next. Because this choice is subjective and infeasible to capture with an unsupervised method.

In this paper, we address the problem of {\em generic guidance} in spatiotemporal data: ``what is the process of guiding analysts in iterative analysis steps on any spatiotemporal dataset?'' In other words, we are interested in an approach which highlights a set of $k$ points that the analyst should consider in the next analysis iteration. This should not be a heuristic-based data-dependent highlighting, but a generic approach which is applied on any spatiotemporal dataset. We describe the desiderata of generic guidance approach as follows.

\vspace{5pt}
\noindent {\bf D1. Genericness.} The guidance component should be agnostic (making no assumption) about the dataset type, attributes and distribution.

\vspace{5pt}
\noindent {\bf D2. Limited Options.} The set of $k$ highlighted points should not be very large because too many options distract the analyst. % \cite{miller1956human}.

\vspace{5pt}
\noindent {\bf D3. Relevance.} The fundamental difference between highlighting and $k$-NN spatial queries \cite{aly2015spatial} is that, in the former, the focus is on $k$ points which have similar characteristics to $p$, hence relevant.
% In other words, we are interested in points which are {\em relevant} to a given point of interest.
For instance, consider a taxi ride in New York for a young male customer for an itinerary of 10 kilometers and \$3 tip. In contrary to thousands of kilometers of geographical distance, the ride is very similar to another one in San Fransisco for a middle-age male customer for an itinerary of 8 kilometers and \$2.5 tip.
% Relevance is a pairwise metric which is associated to point characteristics.
Given two points $p$ and $p'$, we define {\em relevance} as follows.

% \begin{definition}[Relevance]
% Given two points $p$ and $p'$ and their attribute values ${\cal A}_{p}$ and ${\cal A}_{p'}$, the relevance between $p$ and $p'$ is a value between $0$ and $1$ denoted as $\mathit{relevance}(p,p') = \mathit{average}_{a \in {\cal A}_{p} \cup {\cal A}_{p'}}(\mathit{sim({\cal A}_{p}, {\cal A}_{p'}, a)})$.
% \label{def:rel}
% \end{definition}

\begin{dmath}
\label{eq:rel}
\mathit{relevance}(p,p') = \mathit{average}_{a \in {\cal A}_{p} \cup {\cal A}_{p'}}(\mathit{sim(p, p', a)})
\end{dmath}

The similarity function $\mathit{sim}()$ can be any function such as Jaccard and Cosine. Each attribute can have its own similarity function (as string and integer attributes are compared differently.) Then $\mathit{sim}()$ works as an overriding-function which provides encapsulated similarity computations for any type of attribute.

\vspace{5pt}
\noindent {\bf D4. Diversity.} A guidance approach should also consider coverage of all points: $k$ highlighted points should represent distinct regions so that the analyst can observe different aspects of data and decide for the next analysis iteration. Hence, $k$ points should be diverse.
% Diversity is a set-based metric and is associated to geographical distance. We define this metric as follows.
Given a set of points $s = \{ p_1, p_2 \dots \}$, we define {\em diversity} as follows.

\begin{dmath}
\label{eq:divs}
\mathit{diversity}(s) = \mathit{average}_{\{p, p'\} \subseteq s | p \neq p' } \mathit{distance}(p,p')
\end{dmath} 

The function $\mathit{distance}(p,p')$ operates on geographical coordinates of $p$ and $p'$ and can be considered as any distance function of Chebyshev distance family such as Eucledian. However, as distance computations are done in {\em spherical space} using latitude, longitude and altitude, it is au-naturel to employ Harvestine distance shown in Equation \ref{eq:harvestine}.

\begin{dmath}
\label{eq:harvestine}
distance(p,p') = [ acos(cos(p_{lat}) . cos(p'_{lat}) . cos(p_{lng}) . cos(p'_{lng})\\ + cos(p_{lat}) . sin(p'_{lat}). cos(p_{lng}) . sin(p'_{lng}) + sin(p_{lat}) . sin(p'_{lat})) ] \times earth\_radius
\end{dmath}

\noindent {\bf D5. Interactivity.} The exploratory nature of the analysis requires the guidance component to be involved in an interactive process. Hence the analyst can investigate and refine different aspects of spatiotemporal data in iterative steps. For being interactive, the guidance component should be efficient so that the train of thought of analyst would not be broken during the analysis process.

\vspace{5pt}
Following aforementioned desiderata, we forumlate highlighting as an optimization-based problem where we optimize diversity and respect a bound on relevance.

\begin{problem}[\pb]
\label{pb:geoh}
Given an input point $p$ and a threshold $\sigma$, the problem is to return $k$-relevant points to $p$ denoted $S_p$ where $|S_p| = k$ and $\forall p' \in S_p, \mathit{relevance}(p,p') \geq \sigma$ and $\mathit{diversity}(S_p)$ is maximized.
\end{problem}

Problem \ref{pb:geoh} is hard due to the huge space of spatiotemporal data: for any given point $p$, an exhaustive search over all other points is necessary to find $k$ points with maximal relevance. Moreover, the problem expresses interest in obtaining high quality points in two dimensions at the same time (relevance and diversity) which makes the problem more challenging.

% behrooz: talk about quality earlier
\section{Algorithm}
\label{sec:algo}
We propose a solution for \pb\ by inspiring from both recommendation \cite{Omidvar-Tehrani:2015} and
visual highlighting
% \cite{Lohmann:2012,Robinson2011,Liang2010}
\cite{Liang2010,Robinson2011}
methodologies. \pb\ requires an efficient algorithm for dynamically analyzing and comparing geographical points. We propose \framework\ as a solution for the generic guidance problem in spatiotemporal data (Figure \ref{fig:framework}). Although \framework\ operates on points, its functionality can be easily extended to regions using point-clustering methods such as $k$-means.

\framework\ operates in two steps: {\sc Preparation} and {\sc Highlighter}. In order to speed up computing relevance in online execution, we pre-compute an inverted index for each single geographical point in ${\cal P}$ in the offline {\sc Preparation} step (as is commonly done in Web search). Each index ${\cal L}_p$ for the point $p$ stores all other points in ${\cal P}$ in decreasing order of their relevance with $p$. Thanks to the parameter $\sigma$, we only partially materialize the indexes.

% behrooz: explain the whole package

Algorithm \ref{algo:geoh} illustrates the online execution step of \framework\, so called {\sc Highlighter}. The algorithm is a single greedy procedure that solves the \pb\ problem. {\sc Highlighter} is called at each interactive step of \framework\ (as in Figure \ref{fig:framework}). The algorithm admits as input a point $p \in {\cal P}$ and returns the best $k$ points denoted ${\cal S}_p$.

To comply with the desiderata {\bf D5}, we consider a time limit parameter $tlimit$ in Algorithm \ref{algo:geoh}. In a {\em best-effort} strategy, the algorithm bounds user waiting time by $tlimit$ to return the best possible results by then.

\begin{algorithm}[t]
\DontPrintSemicolon
\KwIn{$p \in {\cal P}$, $\sigma$, $k$, $tlimit$}
\KwOut{${\cal S}_p$}
${\cal S}_p \gets get\_top\_k(\mathit{{\cal L}^p})$\;\label{cd:gettopk}
$p_{next} \gets get\_next(\mathit{{\cal L}^p})$\;\label{cd:getnext}
\While{$(tlimit$ $not$ $exceeded \wedge relevance(p,p_{next}) \geq \sigma)$}{\label{cd:beginwhile}
\For{$p_{current} \in {\cal S}_p$}{
\If{$\mathit{diversity\_improved}({\cal S}_p,p_{next},p_{current})$}{\label{cd:betterdiv}
${\cal S}_p \gets \mathit{replace}({\cal S}_p,p_{next},p_{current})$\;
$break$\;
}
}
$p_{next} \gets get\_next({\cal L}^p)$\;}\label{cd:endwhile}
\Return{${\cal S}_p$}\; 
\caption{{\sc Highlighter} Algorithm}
\label{algo:geoh}
\end{algorithm}
% \vspace{-10pt}
% behrooz: mention working of sacrification

{\sc Highlighter} begins by retrieving the most relevant points to $p$ by simply retrieving the $k$ highest ranking points in ${\cal L}_p$ (line \ref{cd:gettopk}). Function $get\_next({\cal L}_p)$ (Line \ref{cd:getnext}) returns the next point $p_{next}$ in ${\cal L}_p$ in sequential order. Lines \ref{cd:beginwhile} to \ref{cd:endwhile} iterate over the inverted indexes to determine if other points should be considered to increase diversity while staying within the time limit and not violating the relevance threshold with the selected point. Since points in ${\cal L}_g$ are sorted on decreasing relevance with $p$, the algorithm can safely stop as soon as the relevance condition is violated (or if the time limit is exceeded).

The algorithm then looks for a candidate point $p_{current} \in {\cal S}_p$ to replace in order to increase diversity. The boolean function $\mathit{diversity\_improved}()$ (line \ref{cd:betterdiv}) checks if by replacing $p_{current}$ by $p_{next}$ in ${\cal S}_p$, the overall diversity of the new ${\cal S}_p$ increases.

% \vspace{5pt}
% \noindent {\bf Complexity Analysis.} The number of diversity improvement loops (lines \ref{cd:beginwhile} to \ref{cd:endwhile}) is $|{\cal L}_p| = |{\cal P}|$ in worst case. For each point $g_{current} \in {\cal S}_p$, we verify if the diversity score is improved by $\mathit{diversity\_improved}()$, hence $\mathcal{O}(k^2$). The time complexity of the algorithm is then $\mathcal{O}(k^2.|{\cal P}|)$.



\section{Scenarios}\label{sec:scenarios}

An analyst has different goals in mind once she needs analyzing data.
Considering the spatiotemporal dataset types aforementioned, we illustrate some
analyst's needs and challenges with the following two realistic scenarios.   
\bigskip

\textit{Scenario 1 - Taxi dataset:} Lucas is a taxi driver which works in New
York city. He wants to see if it is possible to improve his financial revenues
during the week. One option to Lucas may be to choose the potential better
points [$p_1, p_2, \ldots, p_n $] in a neighborhood [$n_1$] to stay considering
a week day and the time of work. He also wants to verify, once he dropoff an
client, which are the closest points [$cp_1, cp_2, \ldots, cp_n $] that can
offer him a new potential trip trajet back to [$n1$]. For instance, Lucas
finishes a trip trajet from a point $p_1$ in a neighborhood [$n_1$] to a point
$p_2$ in a neighborhood $n_2$.  So, given the point p2 in n2, which are the
potential closest points [$cp_1, cp_2, \ldots, cp_n $] (considering the
day/hours of the dropoff) that can bring him back to [$n_1$] with a new client
in a new trip with the small possible time of waiting.
\bigskip

\textit{Scenario 2 - Flight dataset}: Shadi is a flight analyst and she has to
propose new solutions for improve airport and flights performances. She wants to know:
(i) the best time of day/day of week/time of year to fly to minimise delays;
(ii) How does the number of people flying between different locations change
over time? And (iii) Is it possible to detect cascading failures as delays in
one airport create delays in others? And considering this fact, propose a
solution previously. In Shadi's working days she needs to propose quickly
solutions considering the mentioned aspects. 

% Table \ref{table:1} shows the operations for both scenarios.
% 
% \begin{table}[h!]
% \centering
% \begin{tabular}{||c c c ||} 
%  \hline
%  Col1 & Taxi scenario & Taxi scenario  \\ [0.5ex] 
%  \hline\hline
%  1 & 6 & 87837  \\ 
%  2 & 7 & 78  \\
%  3 & 545 & 778 \\
%  4 & 545 & 18744  \\
% 
%  \hline
% \end{tabular}
% \caption{Scenarios}
% \label{table:1}
% \end{table}



\section{Related Work}\label{sec:related-works}

In this paper, we addressed the problem of highlighting geographical points to
guide analysts in consecutives steps. To the best of our knowledge, this is the
first work which formulates the geo-highlighting problem for recommending on
maps. However, our problem relates to a number of others as follows.     

Visual Highlighting. There has been efforts to highlight some pieces of
information in the huge heterogeneous data space so that the analyst can focus
on important aspects \cite{Liang2010,Lohmann:2012,Robinson2011}. Visual
Story Construction is another domain of work \cite{Segel:2010,Samet:2014} which
consecutive analysis steps are created automatically. However, all such highlighting methods are objective and does not consider analyst
interest. In GeoHiglight, we provide a simple yet effective feedback model which
can feed the recommendation algorithm to produce relevant results to current
investigations.        

Some other works have exploited highlighting as a technique to synchronize
coordinated views [CITE CROSSFILTER] or simplify complicated dataset
visualizations \cite{Robinson2011,Alper:2011}. Also in \cite{Philipsen}
different highlighting methods are compared in terms of efficiency and usefulness. Such methods are complementary to ours.     

Spatiotemporal Interactive Analysis and Visualization. Being an interactive
system, it should be efficient and capture user feedback and adapt the utility
function. In terms of efficiency, SpatialHadoop \cite{} and GeoSpark \cite{}
extend Hadoop and Spark ecosystems respectively to boost geographical
computations and visualizations. Such systems can be exploited as the backbone
of GeoHighlight once large-scale data needs to be analyzed.      

In terms of feedback, many off-the-shelf products such as Tableau \cite{} and
RapidMiner \cite{} are designed to visualize different kinds of datasets including
spatiotemporal ones. However, ``recommendation'' is the missing component in
most of such tools: providing a full package of operations and actions, the analyst
may know what to do next. In such system, analysis is usually considered as a
one-shot scenario, once in reality it happens in consecutive steps following
user feedback.        

Recommendation. There exist a huge body of work in recommendation
\cite{Adomavicius:2005} for various domains and datasets. However, spatial
dimension is still untouched and has not received a lot of attention. In
\cite{ChirigatiDDF16} a prediction algorithm for urban data is introduced which operates offline. In
\cite{Levandoski:2012,Magdy2014,HendawiKRBTA15a,Bao2015,Magdy:2014},
recommendation and visualization tools are introduced in specific domains.
However, most the such algorithms are not context-aware (based on user choices)
and does not consider diversity as a global metric.    

\section{Conclusions}\label{sec:conclusions}


\bibliographystyle{abbrv}
\bibliography{main} 

\end{document}
