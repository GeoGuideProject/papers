\section{Illustrative Scenario}\label{sec:scenarios}

An analyst has different goals in mind once she needs analyzing data.
Considering the spatiotemporal dataset types aforementioned, we illustrate some
analyst's needs and challenges with the following two realistic scenarios.   
\bigskip

\textit{Scenario 1 - Taxi dataset:} Lucas is a taxi driver which works in New
York city. He wants to see if it is possible to improve his financial revenues
during the week. One option to Lucas may be to choose the potential better
points [$p_1, p_2, \ldots, p_n $] in a neighborhood [$n_1$] to stay considering
a week day and the time of work. He also wants to verify, once he dropoff an
client, which are the closest points [$cp_1, cp_2, \ldots, cp_n $] that can
offer him a new potential trip trajet back to [$n1$]. For instance, Lucas
finishes a trip trajet from a point $p_1$ in a neighborhood [$n_1$] to a point
$p_2$ in a neighborhood $n_2$.  So, given the point p2 in n2, which are the
potential closest points [$cp_1, cp_2, \ldots, cp_n $] (considering the
day/hours of the dropoff) that can bring him back to [$n_1$] with a new client
in a new trip with the small possible time of waiting.
\bigskip

% Table \ref{table:1} shows the operations for both scenarios.
% 
% \begin{table}[h!]
% \centering
% \begin{tabular}{||c c c ||} 
%  \hline
%  Col1 & Taxi scenario & Taxi scenario  \\ [0.5ex] 
%  \hline\hline
%  1 & 6 & 87837  \\ 
%  2 & 7 & 78  \\
%  3 & 545 & 778 \\
%  4 & 545 & 18744  \\
% 
%  \hline
% \end{tabular}
% \caption{Scenarios}
% \label{table:1}
% \end{table}