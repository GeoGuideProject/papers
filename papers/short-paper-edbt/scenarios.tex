\section{Scenarios}\label{sec:scenarios}

An analyst has different goals in mind, once he needs analyzing data. Considering the spatiotemporal dataset types aforementioned, we illustrate some analyst needs and challenges with the following two realistic scenarios.

\textit{Scenario 1 - (Taxi dataset)}: Lucas is a taxi driver which works in New York city and wants to see if it is possible to improve his financial revenues during the week. He needs to choose the potential better points <p1, p2, ..., pn> in a neighborhood n1 to stay considering each weekday and hour of work. He also wants to verify, once he dropoff an client, which are the closest points that can offer him a new potential trip traject back to n1. For instance, Lucas finishes a trip traject from a point p1 in a neighborhood n1 to a point p2 in a neighborhood n2.  So, given the point p2 in n2, which are the potential closest points <p1, p2, ..., pn> (considering the day/hours of the dropoff) that can bring him back to n1 with a new client in a new trip.    

\textit{Scenario 2 (Flight dataset)}: Shadi is a flight analyst and she has to propose new solutions for improve airport and flights performances. She wants to know: (i) the best time of day/day of week/time of year to fly to minimise delays; (ii) How does the number of people flying between different locations change over time? And (iii) Is it possible to detect cascading failures as delays in one airport create delays in others? And considering this, propose a solution previously. 
